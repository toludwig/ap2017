\documentclass{article}
\usepackage[utf8]{inputenc}
\usepackage{listings}
\usepackage{draftwatermark}
\usepackage{gensymb}
\usepackage{graphicx}
\graphicspath{ {images/} }
\SetWatermarkText{$\lambda$}
\SetWatermarkScale{9}

\title{Advanced Programming 2017\\Assignment 1}
\author{
Tom Jager\\
\texttt{dgr418@alumni.ku.dk}
\and
Tobias Ludwig\\
\texttt{fqj315@alumni.ku.dk}}

\begin{document}

\maketitle

\section{Defining SubsM \textellipsis}

\subsection{\textellipsis to be a Functor}

\begin{lstlisting}[language=Haskell]
instance Functor SubsM where
  fmap f fra = SubsM (\c -> case runSubsM fra c of
    (Left e)        -> Left e
    (Right (a,env)) -> Right (f a,env))
\end{lstlisting}



\subsection{\textellipsis to be an Apllicative}
\begin{lstlisting}[language=Haskell]
instance Applicative SubsM where
  pure a = return a
  ff <*> fa = SubsM (\c0 -> case runSubsM ff c0 of
    (Left e)         -> Left e
    (Right (f, env)) -> case runSubsM fa (env,snd c0) of
      (Left e')         -> Left e'
      (Right (a, env')) -> Right (f a, env'))
\end{lstlisting}


\subsection{\textellipsis to be a Monad}
\begin{lstlisting}[language=Haskell]
instance Monad SubsM where
  return x = SubsM (\c -> Right (x, fst c))
  m >>= f  = SubsM (\c0 -> case runSubsM m c0 of
     (Left e)        -> Left e
     (Right (x,env)) -> case runSubsM (f x) (env,snd c0) of
       (Left e')         -> Left e'
       (Right (x',env')) -> Right (x',env'))
  fail s = error s
\end{lstlisting}


\section{Implementing the Primitives}

Primitives are functions from a list of values to an \texttt{Either} type.
For each operator we generate Errors if the arguments supplied do not match the expected number or types.

\paragraph{strictEquals} is strict because it ensures that same types are compared and no coercion works here.
We do this by checking the types of the arguments and their values. A special case is the \texttt{UndefinedVal}
which we consider to be equal to nothing, not even another \texttt{UndefinedVal}, thus returning always \texttt{FalseVal}.

\paragraph{lessThan} works for two \texttt{IntVal}s or two \texttt{StringVal}s (here lexicographically).

\paragraph{plus} is polymorphic for \texttt{IntVal}s and \texttt{StringVal}s like described.

\paragraph{mult and sub} only work on \texttt{IntVal}s.

\paragraph{mod} catches the special case of ``mod by zero'' which is not defined, thus returning an error.


\section{Utility functions for working with the context}



\end{document}